% Copyright (c) 2004-2006 Marc Fonvieille
% All rights reserved.
%
% Redistribution and use in source and binary forms, with or without
% modification, are permitted provided that the following conditions
% are met:
% 1. Redistributions of source code must retain the above copyright
%    notice, this list of conditions and the following disclaimer.
% 2. Redistributions in binary form must reproduce the above copyright
%    notice, this list of conditions and the following disclaimer in the
%    documentation and/or other materials provided with the distribution.
%
% THIS SOFTWARE IS PROVIDED BY THE AUTHOR AND CONTRIBUTORS ``AS IS'' AND
% ANY EXPRESS OR IMPLIED WARRANTIES, INCLUDING, BUT NOT LIMITED TO, THE
% IMPLIED WARRANTIES OF MERCHANTABILITY AND FITNESS FOR A PARTICULAR PURPOSE
% ARE DISCLAIMED.  IN NO EVENT SHALL THE AUTHOR OR CONTRIBUTORS BE LIABLE
% FOR ANY DIRECT, INDIRECT, INCIDENTAL, SPECIAL, EXEMPLARY, OR CONSEQUENTIAL
% DAMAGES (INCLUDING, BUT NOT LIMITED TO, PROCUREMENT OF SUBSTITUTE GOODS
% OR SERVICES; LOSS OF USE, DATA, OR PROFITS; OR BUSINESS INTERRUPTION)
% HOWEVER CAUSED AND ON ANY THEORY OF LIABILITY, WHETHER IN CONTRACT, STRICT
% LIABILITY, OR TORT (INCLUDING NEGLIGENCE OR OTHERWISE) ARISING IN ANY WAY
% OUT OF THE USE OF THIS SOFTWARE, EVEN IF ADVISED OF THE POSSIBILITY OF
% SUCH DAMAGE.
%
% $FreeBSD$
%
% FreeBSD Flyer
% Use make FORMAT (with FORMAT: pdf, ps or dvi) to build the flyer.
% Two layouts are available: one using Beastie the other one using the
% FreeBSD Logo.  The layout selection is done below via the value of
% the \logo variable.  By default the Beastie layout is enabled.
%
\documentclass[11pt]{article}
\usepackage[T1]{fontenc}
\usepackage[latin1]{inputenc}
% Use the right language
\usepackage[english]{babel}
\usepackage{pslatex}
\usepackage{graphicx}
\usepackage{fancybox}
\usepackage{url}
% Use the right papersize, do not forget to change also the Makefile
\usepackage[verbose,a4paper,noheadfoot,margin=1cm]{geometry}

\usepackage{ifthen}
% Use of the Logo (set the \logo variable below to true) or Beastie
% (\logo variable set to false).
\newcommand{\logo}{false}

% Colors settings
\usepackage{color}
\ifthenelse{\equal{\logo}{true}}{
\definecolor{bkgrdtitle}{rgb}{.69,0,0}
\definecolor{redtitle}{rgb}{.65,.16,.22}
\definecolor{ovalboxcolor}{rgb}{.69,0,0}
}
{
\definecolor{bkgrdtitle}{rgb}{1,.84,.22}
\definecolor{redtitle}{rgb}{.65,.16,.22}
\definecolor{ovalboxcolor}{rgb}{.65,.16,.22}
}

% Some macros
\ifthenelse{\equal{\logo}{true}}{
\newcommand{\titledframe}[3]{%
\boxput*(0,1){\colorbox{bkgrdtitle}{\color{white} \large{\textbf{\textsf{#1}}}}} {\setlength {\fboxsep}{12pt} \color{ovalboxcolor}\Ovalbox {\color{black}\begin{minipage}{#3}#2\end{minipage}}}
}}
{
\newcommand{\titledframe}[3]{%
\boxput*(0,1){\colorbox{bkgrdtitle}{\color{black} \large{\textbf{\textsf{#1}}}}} {\setlength {\fboxsep}{12pt} \color{ovalboxcolor}\Ovalbox {\color{black}\begin{minipage}{#3}#2\end{minipage}}}
}}

\newcommand{\reg}{$^{\mbox{\tiny \textregistered}}$}
\newcommand{\tm}{$^{\mbox{\tiny TM}}$}

\newenvironment{itemizeflyer}%
{ \begin{list}%
	{\textendash}%
	{ \setlength{\leftmargin}{5pt}%
	  \setlength{\itemsep}{0pt}%
	  \setlength{\parskip}{0pt}%
	  \setlength{\parsep}{0pt}}}
{ \end{list}}

\pagestyle{empty}

\begin{document}

\begin{center}
\ifthenelse{\equal{\logo}{true}}{
\includegraphics[scale=0.5]{logo-full.eps}
\vspace{1mm}
}
{
\fontsize{40}{36}\selectfont
{\color{redtitle} \textrm{\textbf{FreeBSD}}}\medskip}
\end{center}
%\vspace{2mm}

% Main part
\begin{center}
\ifthenelse{\equal{\logo}{true}}{
\newcommand{\size}{17.3cm}
}
{\newcommand{\size}{12.7cm}}

\titledframe{What is FreeBSD?}{
FreeBSD is an advanced operating system for x86 compatible (including
Pentium\reg and Athlon\tm), amd64 compatible (including Opteron\tm,
Athlon\tm 64, and EM64T), UltraSPARC\reg, IA-64 (Intel\reg Itanium\reg
Processor Family), PC-98 and ARM architectures.

FreeBSD is derived from BSD, the version of UNIX\reg
developed at the University of California, Berkeley.
}{\size}
\ifthenelse{\equal{\logo}{false}}{
\begin{minipage}{4cm}
\includegraphics[scale=0.3]{../../share/images/flyer/beastie.eps}
\end{minipage}
}

\ifthenelse{\equal{\logo}{true}}{
\vspace{5mm}}{\vspace{3mm}}

\titledframe{Cutting edge features}{
FreeBSD offers advanced networking, performance, security and
compatibility features which are still missing in other
operating systems, even some of the best commercial ones.
}{5cm}
\titledframe{\textsf{\textbf{Powerful Internet solutions}}}{
FreeBSD includes what many consider the reference implementation for
TCP/IP software, the 4.4BSD TCP/IP protocol stack, thereby making it
ideal for network applications and the Internet.
FreeBSD makes an ideal Internet or Intranet server. It provides robust
network services under the heaviest loads and uses memory efficiently
to maintain good response times for thousands of simultaneous user
processes.
}{11.12cm}

\vspace{5mm}

\titledframe{Run a huge number of applications!}{
FreeBSD comes with over 14000 third party software packages
ready to be installed including: Apache, Samba, MySQL\reg,
OpenOffice.org, KDE, GNOME, MPlayer, etc.
\medskip

FreeBSD provides the ports system which fetches sources
from the Internet or from a CD-ROM and compiles binaries, including
building and installing any dependencies.  All these operations are
done in a transparent way for the user.
\medskip

Most of GNU/Linux binaries run under FreeBSD, for example:
Acrobat Reader\reg, Oracle\reg, SAP/R3\reg, Mathematica\reg, Quake3,
etc., without any noticeable speed degradation.
}{17.3cm}

\vspace{5mm}

\titledframe{FreeBSD is easy to install}{
FreeBSD can be installed from a variety of media including CD-ROM,
DVD-ROM, floppy disk, magnetic tape, an MS-DOS\reg
partition, or if you have a network connection, you can install it
directly using anonymous FTP or NFS.
}{17.3cm}

\vspace{5mm}

\titledframe{FreeBSD is free}{
FreeBSD is available free of charge and comes with full source code.
The majority of the source code of the FreeBSD system is
available under the standard BSD license.  In contrast to the GPL
license used by the Linux kernel, the BSD license allows distribution
of derived work without accompanying source code.  This lets companies
use the FreeBSD code as the base of a proprietary product, often
leading to parts of what they do being released back to the community.
}{17.3cm}

\vspace{5mm}

\titledframe{Can handle nearly any task}{
FreeBSD is well-suited for a great number of desktop, server, embedded
or appliance applications.
Today, FreeBSD is not just a server operating system but
aims itself also at end users, particularly newcomers from Windows\reg
and GNU/Linux.
}{17.3cm}

\vspace{5mm}

\titledframe{Contacts}{
\begin{itemizeflyer}
\item Web site: \url{http://www.FreeBSD.org/}
\item FreeBSD Handbook: \url{http://www.FreeBSD.org/doc/handbook/}
\end {itemizeflyer}
}{17.3cm}
\end{center}
%
% Copyrights
\begin{center}
\tiny \copyright 2004-2006 The FreeBSD Project\\
\ifthenelse{\equal{\logo}{true}}{
FreeBSD and the FreeBSD Logo are registered trademarks of the FreeBSD
Foundation.\\}
{FreeBSD is a registered trademark of the FreeBSD Foundation.\\}
All other company and product names may be trademarks of their
respective companies.\\
\ifthenelse{\equal{\logo}{false}}{
BSD Daemon, \copyright 1988 by Marshall Kirk McKusick.  All Rights
Reserved.}{}
\end{center}

\end{document}
