% Copyright (c) 2004 Marc Fonvieille
% All rights reserved.
%
% Redistribution and use in source and binary forms, with or without
% modification, are permitted provided that the following conditions
% are met:
% 1. Redistributions of source code must retain the above copyright
%    notice, this list of conditions and the following disclaimer.
% 2. Redistributions in binary form must reproduce the above copyright
%    notice, this list of conditions and the following disclaimer in the
%    documentation and/or other materials provided with the distribution.
%
% THIS SOFTWARE IS PROVIDED BY THE AUTHOR AND CONTRIBUTORS ``AS IS'' AND
% ANY EXPRESS OR IMPLIED WARRANTIES, INCLUDING, BUT NOT LIMITED TO, THE
% IMPLIED WARRANTIES OF MERCHANTABILITY AND FITNESS FOR A PARTICULAR PURPOSE
% ARE DISCLAIMED.  IN NO EVENT SHALL THE AUTHOR OR CONTRIBUTORS BE LIABLE
% FOR ANY DIRECT, INDIRECT, INCIDENTAL, SPECIAL, EXEMPLARY, OR CONSEQUENTIAL
% DAMAGES (INCLUDING, BUT NOT LIMITED TO, PROCUREMENT OF SUBSTITUTE GOODS
% OR SERVICES; LOSS OF USE, DATA, OR PROFITS; OR BUSINESS INTERRUPTION)
% HOWEVER CAUSED AND ON ANY THEORY OF LIABILITY, WHETHER IN CONTRACT, STRICT
% LIABILITY, OR TORT (INCLUDING NEGLIGENCE OR OTHERWISE) ARISING IN ANY WAY
% OUT OF THE USE OF THIS SOFTWARE, EVEN IF ADVISED OF THE POSSIBILITY OF
% SUCH DAMAGE.
%
% $FreeBSD$
% $FreeBSDde: de-docproj/flyer/flyer.tex,v 1.4 2005/06/19 09:33:44 brueffer Exp $
% basiert auf: 1.10
%
% FreeBSD Flyer
% Use make FORMAT (with FORMAT: pdf, ps or dvi) to build the flyer.
%
\documentclass[11pt]{article}
\usepackage[T1]{fontenc}
\usepackage[latin1]{inputenc}
% Use the right language
\usepackage[german]{babel}
\usepackage{pslatex}
\usepackage{graphicx}
\usepackage{fancybox}
\usepackage{url}
% Use the right papersize, do not forget to change also the Makefile
\usepackage[verbose,a4paper,noheadfoot,margin=1cm]{geometry}
% Colors settings
\usepackage{color}
\definecolor{bkgrdtitle}{rgb}{1,.84,.22}
%\definecolor{bkgrdtitle}{rgb}{1,.81,.3}
%\definecolor{bkgrdtitle}{rgb}{1,.87,.32}
%\definecolor{redtitle}{rgb}{.82,0,0}
%\definecolor{redtitle}{rgb}{.7,.7,.9}
%\definecolor{redtitle}{rgb}{.6,0,0}
%\definecolor{redtitle}{rgb}{.4,0,0}
\definecolor{redtitle}{rgb}{.65,.16,.22}
\definecolor{ovalboxcolor}{rgb}{.65,.16,.22}

% Some macros
\newcommand{\titledframe}[3]{%
\boxput*(0,1){\colorbox{bkgrdtitle}{\color{black} \large{\textbf{\textsf{#1}}}}} {\setlength {\fboxsep}{12pt} \color{ovalboxcolor}\Ovalbox {\color{black}\begin{minipage}{#3}#2\end{minipage}}}
}

\newcommand{\reg}{$^{\mbox{\tiny \textregistered}}$}
\newcommand{\tm}{$^{\mbox{\tiny TM}}$}

\newenvironment{itemizeflyer}%
{ \begin{list}%
	{\textendash}%
	{ \setlength{\leftmargin}{5pt}%
	  \setlength{\itemsep}{0pt}%
	  \setlength{\parskip}{0pt}%
	  \setlength{\parsep}{0pt}}}
{ \end{list}}

\pagestyle{empty}

\begin{document}

\begin{center}
\fontsize{40}{36}\selectfont
{\color{redtitle} \textrm{\textbf{FreeBSD}}}
\end{center}
%\vspace{2mm}

% Main part
\begin{center}
\titledframe{Was ist FreeBSD?}{
FreeBSD ist ein modernes Betriebssystem f�r x86 kompatible (einschlie�lich
Pentium\reg und Athlon\tm), amd64 kompatible (einschlie�lich Opteron\tm,
Athlon\tm 64 und EM64T), Alpha/AXP, IA-64 (Intel\reg Itanium\reg
Prozessorfamilie), PC-98 und UltraSPARC\reg-Architekturen.  Portierungen
auf die PowerPC\reg und ARM\reg Architekturen sind in Arbeit.

FreeBSD ist eine Weiterentwicklung von BSD, dem UNIX\reg
Betriebssystem der University of California, Berkeley.
}{12.7cm}
\begin{minipage}{4cm}
\includegraphics[scale=0.3]{../../share/images/flyer/beastie.eps}
\end{minipage}

\vspace{1mm}

\titledframe{Herausragende Funktionen}{
In den Bereichen Netzwerk, Leistungsf�higkeit, Sicherheit und
Kompatibilit�t besitzt FreeBSD heute schon Funktionen, die in
anderen Betriebssystemen, selbst in den besten kommerziellen,
fehlen.
}{5cm}
\titledframe{\textsf{\textbf{Leistungsf�hige Internet-Dienste}}}{
FreeBSD enth�lt eine von vielen Leuten als Referenzimplementierung
betrachtete TCP/IP Software, den 4.4BSD TCP/IP Protokoll-Stack.
Es ist daher ideal f�r Netzwerkanwendungen und das Internet.
FreeBSD ist bestens geeignet f�r Internet- oder Intranet- Server.
Auch unter h�chsten Lasten arbeiten die Netzwerkdienste zuverl�ssig.
Der effiziente Umgang mit dem Speicher garantiert schnelle Antwortzeiten
f�r tausende gleichzeitig laufende Benutzerprozesse.
}{11cm}

\vspace{5mm}

\titledframe{W�hlen Sie aus einer Vielzahl von Anwendungen!}{
FreeBSD wird mit mehr als 13000 installationsfertigen Softwarepaketen
ausgeliefert, unter anderem: Apache, Samba, MySQL\reg,
OpenOffice.org, KDE, GNOME, MPlayer, etc.\\

FreeBSD bietet die Ports Sammlung.  Dies ist ein System, das Quellcode
aus dem Internet, oder von einer CD-ROM, herunterl�dt und daraus
Bin�rdateien kompiliert, inklusive dem Bau und der Installation von
abh�ngigen Paketen.  Alle Operationen sind dabei f�r den Benutzer
transparent.

Die meisten GNU/Linux Applikationen sind auch unter FreeBSD lauff�hig,
zum Beispiel: Acrobat Reader\reg, Oracle\reg, SAP/R3\reg, Mathematica\reg,
Quake3, etc.  Diese laufen ohne merkliche Geschwindigkeitsverluste.
}{17.3cm}

\vspace{5mm}

\titledframe{FreeBSD ist einfach zu installieren}{
FreeBSD kann von verschiedenen Medien installiert werden, beispielsweise
CD-ROM, DVD-ROM, Disketten, Bandlaufwerken, MS-DOS\reg Partitionen,
oder, wenn eine Netzwerkverbindung verf�gbar ist, per anonymem FTP
oder NFS.
}{17.3cm}

\vspace{5mm}

\titledframe{FreeBSD ist frei}{
FreeBSD ist kostenlos verf�gbar und wird mit vollst�ndigem Quellcode
ausgeliefert.  Der Gro�teil dieses Quellcodes steht unter der BSD
Lizenz.  Im Gegensatz zur GPL Lizenz des Linux Kernels erlaubt es
die BSD Lizenz, ver�nderte Versionen des Originalquellcodes als
Bin�rdateien zu verbreiten, ohne den Quellcode mitliefern zu m�ssen.
Dies f�hrt dazu, dass Firmen den FreeBSD Quellcode als Basis f�r ihre
Produkte nutzen.  Oft flie�en dabei Teile des weiterentwickelten
Codes zur�ck an die Gemeinschaft.
}{17.3cm}

\vspace{5mm}

\titledframe{F�r nahezu jede Aufgabe geeignet}{
FreeBSD ist f�r eine gro�e Anzahl von Desktop-, Server, Embedded- und
Applianceanwendungen gut geeignet.
Heutzutage ist FreeBSD nicht nur ein Betriebssystem f�r Server,
sondern bietet sich auch f�r Endnutzer, vor allem Umsteiger von
Windows\reg und GNU/Linux, an.
}{17.3cm}

\vspace{5mm}

\titledframe{Kontakt}{
\begin{itemizeflyer}
\item Webseite: \url{http://www.FreeBSD.org/de/}
\item FreeBSD Handbuch: \url{http://www.FreeBSD.org/de/handbook/}
\end {itemizeflyer}
}{17.3cm}
\end{center}
%
% Copyrights
\begin{center}
\tiny \copyright 2004-2005 Das FreeBSD Projekt\\
FreeBSD ist ein registriertes Markenzeichen der FreeBSD Foundation.\\
Alle anderen Firmen- und Produktnamen k�nnen Markenzeichen der
jeweiligen Firmen sein.\\
BSD Daemon, \copyright 1988 von Marshall Kirk McKusick.  Alle Rechte
vorbehalten.
\end{center}

\end{document}
