% Copyright (c) 2004-2007 Marc Fonvieille
% All rights reserved.
%
% Redistribution and use in source and binary forms, with or without
% modification, are permitted provided that the following conditions
% are met:
% 1. Redistributions of source code must retain the above copyright
%    notice, this list of conditions and the following disclaimer.
% 2. Redistributions in binary form must reproduce the above copyright
%    notice, this list of conditions and the following disclaimer in the
%    documentation and/or other materials provided with the distribution.
%
% THIS SOFTWARE IS PROVIDED BY THE AUTHOR AND CONTRIBUTORS ``AS IS'' AND
% ANY EXPRESS OR IMPLIED WARRANTIES, INCLUDING, BUT NOT LIMITED TO, THE
% IMPLIED WARRANTIES OF MERCHANTABILITY AND FITNESS FOR A PARTICULAR PURPOSE
% ARE DISCLAIMED.  IN NO EVENT SHALL THE AUTHOR OR CONTRIBUTORS BE LIABLE
% FOR ANY DIRECT, INDIRECT, INCIDENTAL, SPECIAL, EXEMPLARY, OR CONSEQUENTIAL
% DAMAGES (INCLUDING, BUT NOT LIMITED TO, PROCUREMENT OF SUBSTITUTE GOODS
% OR SERVICES; LOSS OF USE, DATA, OR PROFITS; OR BUSINESS INTERRUPTION)
% HOWEVER CAUSED AND ON ANY THEORY OF LIABILITY, WHETHER IN CONTRACT, STRICT
% LIABILITY, OR TORT (INCLUDING NEGLIGENCE OR OTHERWISE) ARISING IN ANY WAY
% OUT OF THE USE OF THIS SOFTWARE, EVEN IF ADVISED OF THE POSSIBILITY OF
% SUCH DAMAGE.
%
% $FreeBSD$
% Original revision: 1.16
%
% Volantino di FreeBSD
% Usa make FORMAT (con FORMAT: pdf, ps o dvi) per creare il volantino.
% Sono disponibili due formati: uno con Beastie e l'altro con il logo di
% FreeBSD.  La selezione del formato avviene tramite il valore della
% variabile \logo.  Di default e' abilitato il formato con Beastie.
%
\documentclass[11pt]{article}
\usepackage[T1]{fontenc}
\usepackage[latin1]{inputenc}
% Use the right language
\usepackage[italian]{babel}
\usepackage{pslatex}
\usepackage{graphicx}
\usepackage{fancybox}
\usepackage{url}
% Use the right papersize, do not forget to change also the Makefile
\usepackage[verbose,a4paper,noheadfoot,margin=1cm]{geometry}

\usepackage{ifthen}
% Usa il Logo (metti \logo a true) o Beastie
% (\logo a false).
\newcommand{\logo}{false}

% Colors settings
\usepackage{color}
\ifthenelse{\equal{\logo}{true}}{
\definecolor{bkgrdtitle}{rgb}{.69,0,0}
\definecolor{redtitle}{rgb}{.65,.16,.22}
\definecolor{ovalboxcolor}{rgb}{.69,0,0}
}
{
\definecolor{bkgrdtitle}{rgb}{1,.84,.22}
\definecolor{redtitle}{rgb}{.65,.16,.22}
\definecolor{ovalboxcolor}{rgb}{.65,.16,.22}
}

% Some macros
\ifthenelse{\equal{\logo}{true}}{
\newcommand{\titledframe}[3]{%
\boxput*(0,1){\colorbox{bkgrdtitle}{\color{white} \large{\textbf{\textsf{#1}}}}} {\setlength {\fboxsep}{12pt} \color{ovalboxcolor}\Ovalbox {\color{black}\begin{minipage}{#3}#2\end{minipage}}}
}}
{
\newcommand{\titledframe}[3]{%
\boxput*(0,1){\colorbox{bkgrdtitle}{\color{black} \large{\textbf{\textsf{#1}}}}} {\setlength {\fboxsep}{12pt} \color{ovalboxcolor}\Ovalbox {\color{black}\begin{minipage}{#3}#2\end{minipage}}}
}}

\newcommand{\reg}{$^{\mbox{\tiny \textregistered}}$}
\newcommand{\tm}{$^{\mbox{\tiny TM}}$}

\newenvironment{itemizeflyer}%
{ \begin{list}%
	{\textendash}%
	{ \setlength{\leftmargin}{5pt}%
	  \setlength{\itemsep}{0pt}%
	  \setlength{\parskip}{0pt}%
	  \setlength{\parsep}{0pt}}}
{ \end{list}}

\pagestyle{empty}

\begin{document}

\begin{center}
\ifthenelse{\equal{\logo}{true}}{
\includegraphics[scale=0.5]{logo-full.eps}
\vspace{1mm}
}
{
\fontsize{40}{36}\selectfont
{\color{redtitle} \textrm{\textbf{FreeBSD}}}\medskip}
\end{center}
%\vspace{2mm}

% Main part
\begin{center}
\ifthenelse{\equal{\logo}{true}}{
\newcommand{\size}{17.3cm}
}
{\newcommand{\size}{12.7cm}}

\titledframe{Cos'� FreeBSD?}{
FreeBSD � un sistema operativo avanzato per architetture compatibili x86
(come Pentium\reg e Athlon\tm), amd64 (come Opteron\tm, Athlon\tm 64,
e EM64T), UltraSPARC\reg, IA-64 (Famiglia di Processori Intel\reg Itanium\reg),
PC-98 e ARM.

FreeBSD � derivato da BSD, la versione di UNIX\reg\ sviluppata all'Universit�
della California, Berkeley.
}{\size}
\ifthenelse{\equal{\logo}{false}}{
\begin{minipage}{4cm}
\includegraphics[scale=0.3]{../../share/images/flyer/beastie.eps}
\end{minipage}
}

\ifthenelse{\equal{\logo}{true}}{
\vspace{5mm}}{\vspace{3mm}}

\titledframe{Funzionalit� avanzate}{
FreeBSD offre funzionalit� di networking avanzato, prestazioni, sicurezza e
compatibilit� che ad oggi mancano ancora in altri sistemi operativi, anche in
alcuni di quelli commerciali.
}{4.8cm}
\titledframe{\textsf{\textbf{Soluzioni Internet potenti}}}{
FreeBSD include quella che molti considerano l'implementazione di
riferimento per il software TCP/IP, lo stack 4.4BSD del protocollo TCP/IP,
rendendolo cos� ideale per applicazioni di rete e Internet.
FreeBSD � ideale per un server Internet o Intranet.  Fornisce servizi di rete 
robusti sotto i carichi pi� pesanti e usa la memoria in maniera efficiente per 
mantenere buoni tempi di risposta per migliaia di processi utente simultanei.
}{11.12cm}

\vspace{4mm}

\titledframe{Esegue un numero enorme di applicazioni}{
FreeBSD � dotato di oltre 17000 pacchetti software di terze parti pronti per
essere installati, tra cui: Apache, Samba, MySQL\reg, OpenOffice.org, KDE,
GNOME, MPlayer, ecc.
\medskip

FreeBSD fornisce il sistema dei port che scarica i sorgenti da Internet o da un
CD-ROM e li compila in forma binaria, inclusa l'installazione di ogni
dipendenza.  Tutte queste operazioni sono fatte in modo trasparente per
l'utente.
\medskip

La maggior parte dei binari per GNU/Linux funzionano sotto FreeBSD senza alcuna
perdita visibile di velocit�, ad esempio: Acrobat Reader\reg, Oracle\reg,
SAP/R3\reg, Mathematica\reg, Quake3, ecc.
}{17.3cm}

\vspace{4mm}

\titledframe{FreeBSD � facile da installare}{
FreeBSD pu� essere installato da una variet� di supporti, inclusi CD-ROM,
DVD-ROM, floppy disk, nastri magnetici, partizioni MS-DOS\reg, o, se hai una
connessione di rete, puoi installarlo direttamente tramite FTP anonimo o NFS.
}{17.3cm}

\vspace{4mm}

\titledframe{FreeBSD � libero}{
FreeBSD � disponibile gratuitamente e viene fornito con il codice sorgente
completo.  La maggior parte del codice sorgente del sistema FreeBSD �
disponibile sotto la licenza standard BSD.  A differenza della licenza GPL
usata dal kernel di Linux, la licenza BSD permette la distribuzione di
software derivato senza accompagnamento del codice sorgente.  Questo permette
alle societ� di usare il codice di FreeBSD come base di un prodotto
proprietario, portando spesso come ritorno alla comunit� il rilascio di alcune
parti di quello che fanno.
}{17.3cm}

\vspace{4mm}

\titledframe{Pu� eseguire quasi ogni operazione}{
FreeBSD � adatto per un gran numero di applicazioni desktop, server, embedded
o appliance.  Oggi, FreeBSD non � pi� solo un sistema operativo server ma punta
anche agli utenti finali, specialmente agli utenti che arrivano da
Windows\reg\ e GNU/Linux.
}{17.3cm}

\vspace{4mm}

\titledframe{Contatti}{
\begin{itemizeflyer}
\item Sito Web: \url{http://www.FreeBSD.org/it/}
\item Manuale di FreeBSD: \url{http://www.freebsd.org/doc/it_IT.ISO8859-15/books/handbook/}
\end{itemizeflyer}
}{17.3cm}
\end{center}
%
% Copyrights
\begin{center}
\tiny \copyright 2004-2008 The FreeBSD Project\\
\ifthenelse{\equal{\logo}{true}}{
FreeBSD e il logo di FreeBSD sono marchi registrati della FreeBSD
Foundation.\\}
{FreeBSD � un marchio registrato della FreeBSD Foundation.\\}
Tutti gli altri nomi di societ� e prodotti possono essere marchi delle loro
rispettive societ�.\\
\ifthenelse{\equal{\logo}{false}}{
BSD Daemon, \copyright 1988 by Marshall Kirk McKusick.
Tutti i diritti riservati.}{}
\end{center}

\end{document}

