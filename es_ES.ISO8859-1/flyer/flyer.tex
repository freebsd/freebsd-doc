% Copyright (c) 2004-2008 Marc Fonvieille
% All rights reserved.
%
% Redistribution and use in source and binary forms, with or without
% modification, are permitted provided that the following conditions
% are met:
% 1. Redistributions of source code must retain the above copyright
%    notice, this list of conditions and the following disclaimer.
% 2. Redistributions in binary form must reproduce the above copyright
%    notice, this list of conditions and the following disclaimer in the
%    documentation and/or other materials provided with the distribution.
%
% THIS SOFTWARE IS PROVIDED BY THE AUTHOR AND CONTRIBUTORS ``AS IS'' AND
% ANY EXPRESS OR IMPLIED WARRANTIES, INCLUDING, BUT NOT LIMITED TO, THE
% IMPLIED WARRANTIES OF MERCHANTABILITY AND FITNESS FOR A PARTICULAR PURPOSE
% ARE DISCLAIMED.  IN NO EVENT SHALL THE AUTHOR OR CONTRIBUTORS BE LIABLE
% FOR ANY DIRECT, INDIRECT, INCIDENTAL, SPECIAL, EXEMPLARY, OR CONSEQUENTIAL
% DAMAGES (INCLUDING, BUT NOT LIMITED TO, PROCUREMENT OF SUBSTITUTE GOODS
% OR SERVICES; LOSS OF USE, DATA, OR PROFITS; OR BUSINESS INTERRUPTION)
% HOWEVER CAUSED AND ON ANY THEORY OF LIABILITY, WHETHER IN CONTRACT, STRICT
% LIABILITY, OR TORT (INCLUDING NEGLIGENCE OR OTHERWISE) ARISING IN ANY WAY
% OUT OF THE USE OF THIS SOFTWARE, EVEN IF ADVISED OF THE POSSIBILITY OF
% SUCH DAMAGE.
%
% $FreeBSD$
%
% FreeBSD Flyer
% Use make FORMAT (with FORMAT: pdf, ps or dvi) to build the flyer.
% Two layouts are available: one using Beastie the other one using the
% FreeBSD Logo.  The layout selection is done below via the value of
% the \logo variable.  By default the Beastie layout is enabled.
%
% The FreeBSD Spanish Documentation Project
% Original revision: r1.16
%
\documentclass[11pt]{article}
\usepackage[T1]{fontenc}
\usepackage[latin1]{inputenc}
% Use the right language
\usepackage[spanish]{babel}
\usepackage{pslatex}
\usepackage{graphicx}
\usepackage{fancybox}
\usepackage{url}
% Use the right papersize, do not forget to change also the Makefile
\usepackage[verbose,a4paper,noheadfoot,margin=1cm]{geometry}

\usepackage{ifthen}
% Use of the Logo (set the \logo variable below to true) or Beastie
% (\logo variable set to false).
\newcommand{\logo}{false}

% Colors settings
\usepackage{color}
\ifthenelse{\equal{\logo}{true}}{
\definecolor{bkgrdtitle}{rgb}{.69,0,0}
\definecolor{redtitle}{rgb}{.65,.16,.22}
\definecolor{ovalboxcolor}{rgb}{.69,0,0}
}
{
\definecolor{bkgrdtitle}{rgb}{1,.84,.22}
\definecolor{redtitle}{rgb}{.65,.16,.22}
\definecolor{ovalboxcolor}{rgb}{.65,.16,.22}
}

% Some macros
\ifthenelse{\equal{\logo}{true}}{
\newcommand{\titledframe}[3]{%
\boxput*(0,1){\colorbox{bkgrdtitle}{\color{white} \large{\textbf{\textsf{#1}}}}} {\setlength {\fboxsep}{12pt} \color{ovalboxcolor}\Ovalbox {\color{black}\begin{minipage}{#3}#2\end{minipage}}}
}}
{
\newcommand{\titledframe}[3]{%
\boxput*(0,1){\colorbox{bkgrdtitle}{\color{black} \large{\textbf{\textsf{#1}}}}} {\setlength {\fboxsep}{12pt} \color{ovalboxcolor}\Ovalbox {\color{black}\begin{minipage}{#3}#2\end{minipage}}}
}}

\newcommand{\reg}{$^{\mbox{\tiny \textregistered}}$}
\newcommand{\tm}{$^{\mbox{\tiny TM}}$}

\newenvironment{itemizeflyer}%
{ \begin{list}%
	{\textendash}%
	{ \setlength{\leftmargin}{5pt}%
	  \setlength{\itemsep}{0pt}%
	  \setlength{\parskip}{0pt}%
	  \setlength{\parsep}{0pt}}}
{ \end{list}}

\pagestyle{empty}

\begin{document}

\begin{center}
\ifthenelse{\equal{\logo}{true}}{
\includegraphics[scale=0.5]{logo-full.eps}
\vspace{1mm}
}
{
\fontsize{40}{36}\selectfont
{\color{redtitle} \textrm{\textbf{FreeBSD}}}\medskip}
\end{center}
%\vspace{2mm}

% Main part
\begin{center}
\ifthenelse{\equal{\logo}{true}}{
\newcommand{\size}{17.1cm}
}
{\newcommand{\size}{12.7cm}}

\titledframe{�Qu� es FreeBSD?}{
FreeBSD es un avanzado sistema operativo para arquitecturas x86 compatibles
(incluyendo Pentium\reg y Athlon\tm), amd64 compatibles (incluyendo
Opteron\reg, Athlon\tm64 EM64T), UltraSPARC\reg, IA-64, PC-98 y ARM.
FreeBSD es un derivado de BSD, la versi�n de UNIX\tm desarrollada en la
Universidad de California, Berkeley.
}{\size}
\ifthenelse{\equal{\logo}{false}}{
\begin{minipage}{4cm}
\includegraphics[scale=0.3]{../../share/images/flyer/beastie.eps}
\end{minipage}
}

\ifthenelse{\equal{\logo}{true}}{
\vspace{4mm}}{\vspace{2mm}}

\titledframe{Caracter�sticas innovadoras}{
FreeBSD ofrece altas prestaciones en comunicaciones de red, rendimiento,
seguridad y compatibilidad, todav�a inexistentes en otros sistemas
operativos, incluyendo los comerciales de mayor renombre.
}{5cm}
\titledframe{\textsf{\textbf{Robustez para Internet}}}{
FreeBSD incluye lo que muchos consideran la implementac�n de referencia de
TCP/IP, el stack 4.4BSD del protocolo TCP/IP, haciendolo ideal para
aplicaciones de red e Internet.

FreeBSD es el servidor ideal para servicios de Internet o Intranet.
Proporciona unos servicios de red robustos, incluso en situaciones de
alta carga, haciendo un uso eficaz de la memoria para mantener buenos
tiempos de respuesta con cientos o miles de procesos simult�neos de usuarios.
}{11.12cm}

\vspace{4mm}

\titledframe{Ejecuta una enorme variedad de aplicaciones}{
Existen m�s de 17000 aplicaciones listas para su uso en FreeBSD,
entre las que est�n Apache, Samba, MySQL\reg, OpenOffice.Org,
KDE, GNOME, MPlayer, etc.

FreeBSD dispone del sistema de ports, que descarga los fuentes
desde Internet o desde CD-ROM y compila binarios, incluyendo
cualquier dependencia.  Todas �stas operaciones se efect�an
de un modo transparente para con el usuario.

La mayor�a de los binarios de GNU/Linux funcionan bajo FreeBSD, por
ejemplo: Acrobat Reader\reg, Oracle\reg, SAP/R3\reg, Mathematica\reg,
Quake3, etc... sin p�rdida palpable de velocidad.
}{17.1cm}

\vspace{4mm}

\titledframe{FreeBSD es f�cil de instalar}{
Se puede instalar FreeBSD desde una gran variedad de soportes, incluyendo
CD-ROM, DVD-ROM, disquetes, cintas magn�ticas, una partici�n MS-DOS\reg,
o si se dispone de conexi�n de red se puede instalar directamente mediante
FTP an�nimo o NFS.
}{17.1cm}

\vspace{4mm}

\titledframe{FreeBSD es libre y gratu�to}{
FreeBSD es gratu�to y se distribuye con el c�digo fuente
de todo el sistema. La mayor�a del c�digo fuente est� bajo la licencia
est�ndar BSD. Al contrario de la Licencia GPL que usa el kernel de Linux
la licencia BSD permite la distribuci�n de software derivado sin la
obligaci�n de inclu�r los fuentes. Esto permite a las empresas usar el
c�digo como base de productos propietarios, permitiendo a menudo que parte
de su producto pueda revertir en la comunidad.
}{17.1cm}

\vspace{4mm}

\titledframe{Puede con casi todo}{
FreeBSD es ideal para gran n�mero de aplicaciones de escritorio,
servidor, aplicaciones empotradas y sistemas dom�ticos.
FreeBSD no es s�lamente un sistema operativo para servidores;
ideal para usuarios finales, en especial reci�n llegados desde
Windows\reg y GNU/Linux.
}{17.1cm}

\vspace{4mm}

\titledframe{M�s informaci�n}{
\begin{itemizeflyer}
\item Sitio web: \url{http://www.FreeBSD.org/es/}
\item FreeBSD Handbook: \url{http://www.freebsd.org/doc/es/books/handbook/}
\end {itemizeflyer}
}{17.1cm}
\end{center}
%
% Copyrights
\begin{center}
\tiny \copyright 2004-2008 El Proyecto FreeBSD\\
\ifthenelse{\equal{\logo}{true}}{
FreeBSD y el logo de FreeBSD son marcas registradas de la Fundaci�n
FreeBSD.\\}
{FreeBSD es una marca registrada de la Fundaci�n FreeBSD.\\}
Todos los dem�s nombres de empresas y productos pueden ser marcas registradas
de la empresa correspondiente.\\
\ifthenelse{\equal{\logo}{false}}{
El daemon de BSD, \copyright 1988 Marshall Kirk McKusick.  Todos los derechos
reservados.}{}
\end{center}

\end{document}
