% Copyright (c) 2004 Marc Fonvieille
% All rights reserved.
%
% Redistribution and use in source and binary forms, with or without
% modification, are permitted provided that the following conditions
% are met:
% 1. Redistributions of source code must retain the above copyright
%    notice, this list of conditions and the following disclaimer.
% 2. Redistributions in binary form must reproduce the above copyright
%    notice, this list of conditions and the following disclaimer in the
%    documentation and/or other materials provided with the distribution.
%
% THIS SOFTWARE IS PROVIDED BY THE AUTHOR AND CONTRIBUTORS ``AS IS'' AND
% ANY EXPRESS OR IMPLIED WARRANTIES, INCLUDING, BUT NOT LIMITED TO, THE
% IMPLIED WARRANTIES OF MERCHANTABILITY AND FITNESS FOR A PARTICULAR PURPOSE
% ARE DISCLAIMED.  IN NO EVENT SHALL THE AUTHOR OR CONTRIBUTORS BE LIABLE
% FOR ANY DIRECT, INDIRECT, INCIDENTAL, SPECIAL, EXEMPLARY, OR CONSEQUENTIAL
% DAMAGES (INCLUDING, BUT NOT LIMITED TO, PROCUREMENT OF SUBSTITUTE GOODS
% OR SERVICES; LOSS OF USE, DATA, OR PROFITS; OR BUSINESS INTERRUPTION)
% HOWEVER CAUSED AND ON ANY THEORY OF LIABILITY, WHETHER IN CONTRACT, STRICT
% LIABILITY, OR TORT (INCLUDING NEGLIGENCE OR OTHERWISE) ARISING IN ANY WAY
% OUT OF THE USE OF THIS SOFTWARE, EVEN IF ADVISED OF THE POSSIBILITY OF
% SUCH DAMAGE.
%
% $FreeBSD$ 
%
% FreeBSD Flyer
% Modo de empleo: make FORMATO (donde FORMATO = pdf, ps o dvi) para 
% generar el flyer.
%
\documentclass[11pt]{article}
\usepackage[T1]{fontenc}
\usepackage[latin1]{inputenc}
% Use the right language
\usepackage[spanish]{babel}
\usepackage{pslatex}
\usepackage{graphicx}
\usepackage{fancybox}
\usepackage{url}
% Ajuste el tamanyo correcto de papel y no olvide modificar el Makefile
\usepackage[verbose,a4paper,noheadfoot,margin=1cm]{geometry}
% Configuracion de color
\usepackage{color}
\definecolor{bkgrdtitle}{rgb}{1,.84,.22}
%\definecolor{bkgrdtitle}{rgb}{1,.81,.3}
%\definecolor{bkgrdtitle}{rgb}{1,.87,.32}
%\definecolor{redtitle}{rgb}{.82,0,0}
%\definecolor{redtitle}{rgb}{.7,.7,.9}
%\definecolor{redtitle}{rgb}{.6,0,0}
%\definecolor{redtitle}{rgb}{.4,0,0}
\definecolor{redtitle}{rgb}{.65,.16,.22}
\definecolor{ovalboxcolor}{rgb}{.65,.16,.22}

% Algunas macros
\newcommand{\titledframe}[3]{%
\boxput*(0,1){\colorbox{bkgrdtitle}{\color{black} \large{\textbf{\textsf{#1}}}}} {\setlength {\fboxsep}{12pt} \color{ovalboxcolor}\Ovalbox {\color{black}\begin{minipage}{#3}#2\end{minipage}}}
}

\newcommand{\reg}{$^{\mbox{\tiny \textregistered}}$}
\newcommand{\tm}{$^{\mbox{\tiny TM}}$}

\newenvironment{itemizeflyer}%
{ \begin{list}%
	{\textendash}%
	{ \setlength{\leftmargin}{5pt}%
	  \setlength{\itemsep}{0pt}%
	  \setlength{\parskip}{0pt}%
	  \setlength{\parsep}{0pt}}}
{ \end{list}}

\pagestyle{empty}

\begin{document}

\begin{center}
\fontsize{40}{36}\selectfont
{\color{redtitle} \textrm{\textbf{FreeBSD}}}
\end{center}
%\vspace{2mm}

% Parte principal
\begin{center}
\titledframe{�Qu� es FreeBSD?}{
FreeBSD es un sistema operativo muy avanzado que funciona en 
arquitecturas x86 (Intel\reg y AMD\tm), AMD64 (AMD Opteron\tm
y Athlon\tm 64), DEC Alpha, IA-64
(Familia de Procesadores Intel\reg Itanium\reg), PC-98 y UltraSPARC\reg.
Actualmente est� siendo portado a las arquitecturas PowerPC\reg y 
MIPS.

FreeBSD proviene de BSD, la versi�n de UNIX\reg desarrollada en 
la Universidad de California, en Berkeley.
}{12.7cm}
\begin{minipage}{4cm}
\includegraphics[scale=0.3]{../../share/images/flyer/beastie.eps}
\end{minipage}

\vspace{1mm}

\titledframe{Caracter�sticas innovadoras}{
%\titledframe{Cutting edge features}{
FreeBSD ofrece muy altas prestaciones en comunicaciones 
en red, rendimiento, seguridad y 
caracter�sticas de compatibilidad que a�n no esta otros sistemas 
operativos, incluyendo algunos de los mejores S.O. comerciales.
}{5cm}
\titledframe{\textsf{\textbf{Robustez para Internet}}}{
%\titledframe{\textsf{\textbf{Powerful Internet solutions}}}{
FreeBSD incluye lo que para muchos es la implementaci�n referencial 
del software TCP/IP, la pila TCP/IP de 4.4.BSD, lo que lo convierte 
en ideal para aplicaciones de red e Internet. Dispone de servicios 
de red robustos bajo las cargas m�s grandes y gestiona la memoria 
de manera eficiente para lograr buenos tiempos de respuesta bajo 
miles de procesos de usuario simult�neos.
}{11cm}

\vspace{5mm}

\titledframe{Ejecuta una enorme variedad de aplicaciones}{
%\titledframe{Run a huge number of applications!}{
Existen m�s de 10000 aplicaciones listas para su uso en FreeBSD, 
entre las que est�n Apache, Samba, MySQL\reg, OpenOffice.Org, 
KDE, GNOME, MPlayer, etc.\\
FreeBSD dispone del sistema de ports, que descarga los fuentes 
desde Internet o desde CD-ROM y compila binarios, incluyendo 
cualquier dependencia.  Todas �stas operaciones se efect�an 
de un modo transparente para con el usuario.
La mayor�a de los binarios de GNU/Lunux funcionan bajo FreeBSD, por 
ejemplo: Acrobat Reader\reg, Oracle\reg, SAP/R3\reg, Mathematica\reg,
Quake3, etc... sin p�rdida palpable de velocidad.
}{17.3cm}

\vspace{5mm}

\titledframe{FreeBSD es f�cil de instalar}{
FreeBSD puede instalarse desde una gran variedad de soportes: CD-ROM,
DVD-ROM, disquetes, cinta magn�tica y desde una partici�n MS-DOS\reg, 
o si dispone de una conexi�n de red puede instalarlo directamente 
usando FTP an�nimo o NFS.
}{17.3cm}

\vspace{5mm}

\titledframe{FreeBSD es libre y gratu�to}{
FreeBSD es gratu�to y se distribuye con el c�digo fuente 
de todo el sistema. La mayor�a del c�digo fuente del sistema 
FreeBSD est� bajo la licencia est�ndar BSD. Al contrario de la 
Licencia GPL que usa el kernel de Linux la licencia BSD permite 
la distribuci�n de software derivado sin la obligaci�n de inclu�r 
los fuentes. �sto permite a las empresas usar el c�digo de FreeBSD 
como base de productos propietarios, permitiendo a menudo que parte 
de su producto pueda revertir en la comunidad. 
}{17.3cm}

\vspace{5mm}

\titledframe{Puede con (casi) todo}{
FreeBSD es ideal para gran n�mero de aplicaciones de escritorio, 
servidor, aplicaciones empotradas y sistemas dom�ticos.
FreeBSD no es s�lamente un sistema operativo para servidores, sino 
ideal para usuarios finales, en especial reci�n llegados desde 
Windows\reg y GNU/Linux.
}{17.3cm}

\vspace{5mm}

\titledframe{M�s informaci�n}{
\begin{itemizeflyer}
\item Sitio web: \url{http://www.FreeBSD.org/}
\item FreeBSD Handbook: \url{http://www.FreeBSD.org/doc/handbook/}
\end {itemizeflyer}
}{17.3cm}
\end{center}
%
% Copyrights
\begin{center}
\tiny \copyright 2004 The FreeBSD Project\\
FreeBSD is a registered trademark of Wind River Systems,
Inc. This is expected to change soon.\\
All other company and product names may be trademarks of their
respective companies.\\
BSD Daemon, \copyright 1988 by Marshall Kirk McKusick.  All Rights
Reserved.
\end{center}

\end{document}
