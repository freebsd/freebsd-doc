% Gebaseerd op: 1.9
% Vertaald door: Siebrand Mazeland
%
% Copyright (c) 2004 Marc Fonvieille
% Alle rechten voorbehouden.
%
% Herdistributie en gebruik in vorm van broncode of binaire vorm,
% met of zonder wijzigingen, zijn toegestaan, mits aan de volgende
% voorwaarden is voldaan:
% 1. Bij herdistributie van broncode moet de bovenstaande copyright
%    melding in stand blijven alsmede de onderstaande disclaimer.
% 2. Bij herdistributie in binaire vorm moet de bovenstaande copyright
%    melding worden weergegeven, deze lijst van voorwaarden en de
%    onderstaande disclaimer dient in de documentatie en/of andere
%    materialen die onderdeel uitmaken van de distributie aanwezig te
%    zijn.
%
% THIS SOFTWARE IS PROVIDED BY THE AUTHOR AND CONTRIBUTORS ``AS IS'' AND
% ANY EXPRESS OR IMPLIED WARRANTIES, INCLUDING, BUT NOT LIMITED TO, THE
% IMPLIED WARRANTIES OF MERCHANTABILITY AND FITNESS FOR A PARTICULAR PURPOSE
% ARE DISCLAIMED.  IN NO EVENT SHALL THE AUTHOR OR CONTRIBUTORS BE LIABLE
% FOR ANY DIRECT, INDIRECT, INCIDENTAL, SPECIAL, EXEMPLARY, OR CONSEQUENTIAL
% DAMAGES (INCLUDING, BUT NOT LIMITED TO, PROCUREMENT OF SUBSTITUTE GOODS
% OR SERVICES; LOSS OF USE, DATA, OR PROFITS; OR BUSINESS INTERRUPTION)
% HOWEVER CAUSED AND ON ANY THEORY OF LIABILITY, WHETHER IN CONTRACT, STRICT
% LIABILITY, OR TORT (INCLUDING NEGLIGENCE OR OTHERWISE) ARISING IN ANY WAY
% OUT OF THE USE OF THIS SOFTWARE, EVEN IF ADVISED OF THE POSSIBILITY OF
% SUCH DAMAGE.
%
% $FreeBSD$
%
% FreeBSD Flyer
% Gebruik make FORMAT (waar FORMAT: pdf, ps of dvi) om de flyer te bouwen.
%
\documentclass[11pt]{article}
\usepackage[T1]{fontenc}
\usepackage[latin1]{inputenc}
% Use the right language
\usepackage[dutch]{babel}
\usepackage{pslatex}
\usepackage{graphicx}
\usepackage{fancybox}
\usepackage{url}
% Gebruik het correcte papierformaat, vergeet niet ook de Makefile te
% wijzigen
\usepackage[verbose,a4paper,noheadfoot,margin=1cm]{geometry}
% Kleurinstellingen
\usepackage{color}
\definecolor{bkgrdtitle}{rgb}{1,.84,.22}
%\definecolor{bkgrdtitle}{rgb}{1,.81,.3}
%\definecolor{bkgrdtitle}{rgb}{1,.87,.32}
%\definecolor{redtitle}{rgb}{.82,0,0}
%\definecolor{redtitle}{rgb}{.7,.7,.9}
%\definecolor{redtitle}{rgb}{.6,0,0}
%\definecolor{redtitle}{rgb}{.4,0,0}
\definecolor{redtitle}{rgb}{.65,.16,.22}
\definecolor{ovalboxcolor}{rgb}{.65,.16,.22}

% Een paar macro's
\newcommand{\titledframe}[3]{%
\boxput*(0,1){\colorbox{bkgrdtitle}{\color{black} \large{\textbf{\textsf{#1}}}}} {\setlength {\fboxsep}{12pt} \color{ovalboxcolor}\Ovalbox {\color{black}\begin{minipage}{#3}#2\end{minipage}}}
}

\newcommand{\reg}{$^{\mbox{\tiny \textregistered}}$}
\newcommand{\tm}{$^{\mbox{\tiny TM}}$}

\newenvironment{itemizeflyer}%
{ \begin{list}%
	{\textendash}%
	{ \setlength{\leftmargin}{5pt}%
	  \setlength{\itemsep}{0pt}%
	  \setlength{\parskip}{0pt}%
	  \setlength{\parsep}{0pt}}}
{ \end{list}}

\pagestyle{empty}

\begin{document}

\begin{center}
\fontsize{40}{36}\selectfont
{\color{redtitle} \textrm{\textbf{FreeBSD}}}
\end{center}
%\vspace{2mm}

% Main part
\begin{center}
\titledframe{Wat is FreeBSD?}{
FreeBSD is geavanceerd besturingssysteem voor x86 compatible (inclusief
Pentium\reg en Athlon\tm), amd64 compatibele (inclusief Opteron\tm,
Athlon\tm 64 en EM64T), Alpha/AXP, IA-64 (Intel\reg Itanium\reg
Processor Familie), PC-98 en UltraSPARC\reg architecturen.  Het wordt
nog geporteerd naar PowerPC\reg en MIPS architecturen.

FreeBSD is afgeleid van BSD, de versie van UNIX\reg die is ontwikkeld
door de University of California, Berkeley.
}{12.7cm}
\begin{minipage}{4cm}
\includegraphics[scale=0.3]{../../share/images/flyer/beastie.eps}
\end{minipage}

\vspace{1mm}

\titledframe{De nieuwste mogelijkheden}{
FreeBSD biedt geavanceerde netwerkmogelijkheden, prestaties,
beveiliging en compatibiliteitsmogelijkheden die nog steeds ontbreken
in andere besturingssystemen, zelfs in de beste commerci/edieresisle.
}{5cm}
\titledframe{\textsf{\textbf{Krachtige Internet oplossingen}}}{
FreeBSD bevat wat gezien kan worden als de referentieimplementatie
voor TCP/IP software, de 4.4BSD TCP/IP protocol stack, waardoor het
ideaal is voor netwerktoepassingen en het Internet.
FreeBSD is ideal voor Internet en intranet servers. Het biedt robuuste
netwerkdiensten onder de zwaarste omstandigheden en maakt
effici/edieresisnt gebruik van geheugen om goed te blijven reageren al
draaien er duizenden gebruikersprocessen.
}{11cm}

\vspace{5mm}

\titledframe{Draai een enorm aantal applicaties!}{
In FreeBSD zitten meer dan 12.000 softwarepakketten die zo
ge\"{i}nstalleerd kunnen worden, waaronder: Apache, Samba,
MySQL\reg, OpenOffice.org, KDE, GNOME, MPlayer, enzovoort.\\

FreeBSD kent het portssysteem dat broncode van Internet of een CD-ROM
haalt en daarvan binaire bestanden compileert, inclusief het bouwen en
installeren van bestaande afhankelijkheden. Al deze operaties worden op
een voor de gebruiker transparante wijze uitgevoerd.\\

De meeste GNU/Linux binaire bestanden draaien onder FreeBSD,
bijvoorbeeld: Acrobat Reader\reg, Oracle\reg, SAP/R3\reg,
Mathematica\reg, Quake3, enzovoort, zonder enige merkbare achteruitgang
in snelheid.
}{17.3cm}

\vspace{5mm}

\titledframe{FreeBSD is eenvoudig te installeren}{
FreeBSD kan vanaf vele media ge\"{i}nstalleerd worden, waaronder
CD-ROM, DVD-ROM, floppy disk, magnetische band, een MS-DOS\reg partitie
of als er een netwerkverbinding beschikbaar is, kan er direct vanaf
anonieme FTP of NFS ge\"{i}nstalleerd worden.
}{17.3cm}

\vspace{5mm}

\titledframe{FreeBSD is gratis}{
FreeBSD is gratis beschikbaar en de volledige broncode wordt
bijgeleverd. De meeste broncode van het FreeBSD systeem is beschikbaar
onder de standaard BSD licentie.  In tegenstellingen tot de GPL
licentie die voor de Linux kernel wordt gebruikt, staat de BSD licentie
distributie van afgeleide werken toe zonder dat daarbij de broncode
wordt geleverd. Hierdoor kunnen bedrijven de code van FreeBSD als de
basis van hun gesloten product gebruiken, wat toch nog vaak leidt tot
het teruggeven aan de gemeenschap van dat wat is ontwikkeld.
}{17.3cm}

\vspace{5mm}

\titledframe{Kan vrijwel iedere taak aan}{
FreeBSD is bijzonder geschikt voor een groot aantal bureaublad-,
server-, embedded of appliancetoepassingen. Vandaag de dag is FreeBSD
niet alleen een besturingssysteem voor servers, maar is het ook gericht
op eindgebruikers, in het bijzonder nieuwkomers van Windows\reg en
GNU/Linux.
}{17.3cm}

\vspace{5mm}

\titledframe{Contact}{
\begin{itemizeflyer}
\item Website: \url{http://www.FreeBSD.org/}
\item FreeBSD Handboek: \url{http://www.freebsd.org/doc/nl_NL.ISO8859-1/books/handbook/}
\end {itemizeflyer}
}{17.3cm}
\end{center}
%
% Copyrights
\begin{center}
\tiny \copyright 2004-2005 The FreeBSD Project\\
FreeBSD is een geregistreerd handelsmerk van de FreeBSD Foundation.\\
Alle andere bedrijfs- en productnamen kunnen handelsmerken zijn van hun
respectievelijke bedrijven.\\
BSD Daemon, \copyright 1988 door Marshall Kirk McKusick.  Alle rechten
voorbehouden.
\end{center}

\end{document}
